\documentclass{article}
\usepackage[utf8]{inputenc}
\usepackage{amsmath}
\usepackage{amssymb}
\usepackage{graphicx}

\title{Shi-Tomashi}
\author{filip.reichl, tomas.adam, lucia.szalonova, roman.varga }
\date{December 2019}

\begin{document}

\maketitle

\section{Pseudocode}
\rcomment{Taylor Series:} 
 T(x,y) \approx f(u,v)+(x-u) $f_x$(u,v)+(y-v) $f_y$(u,v)+... \\
 


Rewriting the shifted intesity using the above formula:\\ 
I(x+u, y+v) \approx I(x,y)+ \partial $I_(_x_,_y_)$/ \partial _x$ \hspace{0,1cm} u +  \partial $I_(_x_,_y_)$/ \partial _y$ \hspace{0,1cm} v \\


Let: \partial $I_(_x_,_y_)$/ \partial x = $I_x$\hspace{0,2cm} and 
\hspace{0,2cm}\partial  $I_(_x_,_y_)$/ \partial y = $I_y$ \hfill \medskip \break 
i.e. \hspace{0,2cm} $I_x$ \hspace{0,2cm} and \hspace{0,2cm} $I_y$ \hspace{0,2cm} are \hspace{0,2cm} image \hspace{0,1cm} derivatives  \hspace{0,1cm}
in \hspace{0,1cm} the\hspace{0,1cm} X \hspace{0,1cm}and\hspace{0,1cm} Y \hspace{0,1cm}directions\hspace{0,1cm} respectively. 

then
\break 
\vspace{1cm} 
E(u,v) = \Sigma \hspace{0,1cm} \omega(x,y) \hspace{0,1cm} [I(x,y) + $I_x$  u + $I_y$  v - I(x,y)]^2$  

E(u,v) = \Sigma \hspace{0,1cm} \omega(x,y)\hspace{0,1cm}[$I_x$u + $I_y$v]^2$ \\


Expanding, E(u,v) = \Sigma \hspace{0,1cm} \omega(x,y) \hspace{0,1cm} [$I_x$^2$ u^2$ + $I_y$^2$ * v^2$ + 2 $I_x$ $I_y$ u v] \\

Taking u,v out and re-write in Matrix notation give us:\\

E(u,v) \approx (u,v)M(x y)\\

Here, M = \omega(x,y)* $$
                    \begin{pmatrix}
                    \Sigma \hspace{0,1cm} $I_x$^2 & \Sigma $I_x$ $I_y$\\
                    \Sigma \hspace{0,1cm} $I_x$ $I_y$ & \Sigma $I_y$^2\\
                    \end{pmatrix}
                $$\\
\break
\begin{math}
R= min( \lambda1, \lambda2)

det M \hspace{0,1cm} =\hspace{0,1cm} \lambda1 \hspace{0,1cm} \lambda2

trace M \hspace{0,1cm} =\hspace{0,1cm} \lambda1  + \lambda2
\end{math}
\end{document}
